% !TEX TS-program = xelatex
% !TEX encoding = UTF-8 Unicode
% !Mode:: "TeX:UTF-8"

\documentclass{resume}
\usepackage{zh_CN-Adobefonts_external} % Simplified Chinese Support using external fonts (./fonts/zh_CN-Adobe/)
%\usepackage{zh_CN-Adobefonts_internal} % Simplified Chinese Support using system fonts
\usepackage{linespacing_fix} % disable extra space before next section
\usepackage{cite}
\usepackage{hyperref}

\begin{document}
\pagenumbering{gobble} % suppress displaying page number

\name{陈彦哲}

% {E-mail}{mobilephone}{homepage}
% be careful of _ in emaill address
\contactInfo{yanzhe.cn@gmail.com}{(+86) 185-1669-3610}{}
% {E-mail}{mobilephone}
% keep the last empty braces!
%\contactInfo{xxx@yuanbin.me}{(+86) 131-221-87xxx}{}
 
\section{\faGraduationCap\ 教育背景}
\datedsubsection{\textbf{并行与分布式系统研究所, 上海交通大学}}{2014.9 - 2017.3}
\datedline{\textit{软件工程硕士}, \textit{导师}: \href{http://ipads.se.sjtu.edu.cn/binyu_zang}{\textit{臧斌宇教授}}}{\textit{绩点}: 2.81\ /\ 3.3, \textit{排名}: 2\ /\ 96}
\datedsubsection{\textbf{上海交通大学}}{2010.9 - 2014.6}
\datedline{\textit{软件工程学士}}{\textit{绩点}: 3.79\ /\ 4.3, \textit{排名}: 8\ /\ 99}

\section{\faBriefcase\ 实习经历}
\datedsubsection{\textbf{微软中国}}{ 2016.7 - 2016.9 }
\role{云计算与企业事业部} {云计算创新中心,软件开发工程师}

设计并实现了一套Azure开销监控服务,帮助团队节省支出。

该系统以单页应用(SPA)的形式提供服务,使用MEAN技术栈,支持Docker部署。

\section{\faUsers\ 项目经历}
\datedsubsection{\textbf{DrTM}}{ 2015 - 2016 }
\role{分布式事务处理}{C++,科研项目}
DrTM是一个分布式的内存键值数据库,旨在分布式环境下提供高吞吐量、低延迟的事务支持。
\\[5pt]
DrTM结合硬件事务内存(HTM)和远程直接内存存取(RDMA)两种硬件技术,设计了一套硬件友好的并发控制协议;在该协议的基础上构建了一个分布式内存键值数据库。
基于此项目的两篇论文分别被SOSP’15和EuroSys’16(均为计算机系统领域顶级会议)录用。
\\[5pt]
DrTM是合作项目,我负责的工作有:
\begin{itemize}[parsep=0.5ex]
  \item 基于RDMA设计了一个带租约的共享锁
  \item 设计实现了一套基于乐观并发控制(OCC)的协议
  \item 解决了HTM和RDMA环境下事务难以备份的问题
\end{itemize}

\datedsubsection{\textbf{PowerLyra}}{ 2014 - 2015 }
\role{分布式图计算}{C++,科研项目}
图计算是针对具有复杂依赖关系和迭代计算特征的算法所设计的基于大数据的并行计算模型。PowerLyra 是开源分布式图计算框架 GraphLab 的改进版。
\\[5pt]
PowerLyra 分析了 业界普遍采用的两种图划分方式:Vertex-Cut 和 Edge-Cut 的优劣势, 提出了 Hybrid-Cut 方法,针对不同类型的顶点采用不同的划分策略以及计算方式,减少了冗余消息的传递,有效地提升了性能。
基于此项目的论文被EuroSys’15录用,并获得该届会议的最佳论文奖。
\\[5pt]
PowerLyra是合作项目,我负责的工作有:
\begin{itemize}[parsep=0.5ex]
  \item 分析现有划分方法的冗余因子
  \item 合作实现了PowerLyra中的 Hybrid-Cut 方法
\end{itemize}

\datedsubsection{\textbf{GENE-MAP}}{ 2013 - 2014 }
\role{地图泛化}{C++,比赛项目}
地图泛化是在线地图服务的核心技术之一。GENE-MAP权衡了泛化过程中的精准度和速度,设计了一种高效的泛化方法。
\\[5pt]
GENE-MAP利用迭代和贪心的思想,在精准度和速度间做出权衡,为地图泛化提供了一种解决方案。该项目参加了2014年的GISCUP大赛并获得第三名的成绩,相关论文同时被SIGSPATIAL’14录用。
\\[5pt]
GENE-MAP是个人项目,我完成的工作有:
\begin{itemize}[parsep=0.5ex]
  \item 设计并实现了一种基于迭代和贪心的泛化方法,利用OpenMP实现了算法的并行
\end{itemize}

\section{\faFile\ 论文发表}
\titleformat{\subsection}
  {\normalsize\raggedright}
  {}{0em}
  {}
\datedsubsection{\textbf{Fast and General Distributed Transactions using RDMA and HTM}}{\textbf{EuroSys} 2016}
\datedline{\small\underline{Yanzhe Chen}, Xinda Wei, Jiaxin Shi, Rong Chen and Haibo Chen.}{\href{http://ob88vwut3.bkt.clouddn.com/papers/drtmr-eurosys16.pdf}{\faFilePdfO}\ \ \href{http://ob88vwut3.bkt.clouddn.com/slides/drtmr-eurosys16-slides.pptx}{\faFilePowerpointO}}

\datedsubsection{\textbf{Fast In-memory Transaction Processing using RDMA and HTM}}{\textbf{SOSP} 2015}
\datedline{\small Xingda Wei, Jiaxin Shi, \underline{Yanzhe Chen}, Rong Chen, Haibo Chen.}{\href{http://ob88vwut3.bkt.clouddn.com/papers/drtm-sosp15.pdf}{\faFilePdfO}\ \ \href{http://ob88vwut3.bkt.clouddn.com/slides/drtm-sosp15-slides.pptx}{\faFilePowerpointO}}

\titlespacing*{\subsection}{0cm}{*1.8}{*0.6}
\datedsubsection{\textbf{PowerLyra: Differentiated Graph Computation and Partitioning on Skewed Graphs}}{\textbf{EuroSys} 2015}
\datedline{\small Rong Chen, Jiaxin Shi, \underline{Yanzhe Chen}, Haibo Chen.}{\href{http://ob88vwut3.bkt.clouddn.com/papers/powerlyra-eurosys15.pdf}{\faFilePdfO}\ \ \href{http://ob88vwut3.bkt.clouddn.com/slides/powerlyra-eurosys15-slides.pptx}{\faFilePowerpointO}}

\datedsubsection{\textbf{Greedy Map Generalization by Iterative Point Removal}}{\textbf{SIGSPATIAL} 2014}
\datedline{\small\underline{Yanzhe Chen}, Yin Wang, Rong Chen, Haibo Chen and Binyu Zang.}{\href{http://ob88vwut3.bkt.clouddn.com/papers/gmap-sigspatialcup14.pdf}{\faFilePdfO}\ \ \href{http://ob88vwut3.bkt.clouddn.com/slides/gmap-sigspatialcup14-slides.pptx}{\faFilePowerpointO}}

\section{\faHeartO\ 获奖情况}
\datedsubsection{ACM EuroSys 最佳论文奖}{2015}
\datedsubsection{上海交通大学硕士一等学业奖学金}{2014}
\datedsubsection{ACM SIGSPATIAL GISCUP 第三名}{2014}
\datedsubsection{上海交通大学优秀毕业生}{2014}
\datedsubsection{上海交通大学心动二等奖学金}{2013}
\datedsubsection{新鸿基地产郭氏基金奖助学金}{2012}
\datedsubsection{申银万国奖学金}{2011}

\section{\faBook\ 助教工作}
\datedsubsection{计算机系统设计和实现}{2016}
\datedsubsection{分布式系统}{2015}
\datedsubsection{程序设计}{2013}

\section{\faCogs\ 专业技能}
\begin{itemize}[parsep=1.0ex]
  \item 编程语言: 熟练使用 C++,有 JavaScript, Bash, Java的使用经历
  \item 系统工具: 4年以上的Unix使用经验,了解 Git,Vim等开发工具
  \item 英语水平: CET-6 561分,在国际学术会议上做过两次演讲
\end{itemize}

\end{document}
