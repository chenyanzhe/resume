% !TEX TS-program = xelatex
% !TEX encoding = UTF-8 Unicode
% !Mode:: "TeX:UTF-8"

\documentclass{resume}
\usepackage{zh_CN-Adobefonts_external} % Simplified Chinese Support using external fonts (./fonts/zh_CN-Adobe/)
%\usepackage{zh_CN-Adobefonts_internal} % Simplified Chinese Support using system fonts
\usepackage{linespacing_fix} % disable extra space before next section
\usepackage{cite}
\usepackage{hyperref}

\begin{document}
\pagenumbering{gobble} % suppress displaying page number

\name{陈彦哲}

% {E-mail}{mobilephone}{homepage}
% be careful of _ in emaill address
\contactInfo{yanzhe.cn@gmail.com}{(+86) 185-1669-3610}{}
% {E-mail}{mobilephone}
% keep the last empty braces!
%\contactInfo{xxx@yuanbin.me}{(+86) 131-221-87xxx}{}
 
\section{\faGraduationCap\ 教育背景}
\datedsubsection{\textbf{并行与分布式系统研究所, 上海交通大学}}{2014.9 - 2017.3}
\datedline{\textit{软件工程硕士}, \textit{导师}: \href{http://ipads.se.sjtu.edu.cn/binyu_zang}{\textit{臧斌宇教授}}}{\textit{绩点}: 2.81\ /\ 3.3, \textit{排名}: 2\ /\ 95}
\datedsubsection{\textbf{上海交通大学}}{2010.9 - 2014.6}
\datedline{\textit{软件工程学士}}{\textit{绩点}: 3.79\ /\ 4.3, \textit{排名}: 8\ /\ 99}

\section{\faUsers\ 项目经历}
\datedsubsection{\textbf{DrTM}}{ 2015 - 2016 }
\role{分布式事务处理}{C++,科研项目}
DrTM是一个基于\ \underline{HTM}和\ \underline{RDMA}两种硬件构建的分布式事务处理系统。
\\[2pt]
DrTM利用HTM提供的\ \underline{强原子性}降低并发控制的复杂性,简化分布式事务的设计。
\\[2pt]
DrTM利用RDMA提供的\ \underline{强一致性}降低分布式提交的延迟,提高分布式事务的性能。
\\[2pt]
DrTM是合作项目,我负责的工作有:
\begin{itemize}[parsep=0.5ex]
  \item 设计并实现了一套基于HTM和RDMA的分布式并发控制协议
  \item 实现了一组基准测试集 (TPC-C, SmallBank, TATP) 以验证系统的高性能
  \item 进一步扩展协议,支持事务的备份,提供了系统的持久性
\end{itemize}

\datedsubsection{\textbf{PowerLyra}}{ 2013 - 2015 }
\role{分布式图计算}{C++,科研项目}
自然图中通常存在严重的不均衡性:少量节点度数很高,大量节点度数很低。
\\[2pt]
PowerLyra是针对\ \underline{自然图}而设计的分布式图计算系统。
\\[2pt]
PowerLyra对高度数节点和低度数节点\ \underline{区别对待},提出了混合的划分和计算方法。
\\[2pt]
PowerLyra是合作项目,我负责的工作有:
\begin{itemize}[parsep=0.5ex]
  \item 合作实现了PowerLyra中的混合划分方法
  \item 将此划分方法与已有的方法进行比较,验证了混合的有效性
\end{itemize}

\datedsubsection{\textbf{GENE-MAP}}{ 2013 - 2014 }
\role{地图泛化}{C++,比赛项目}
地图泛化常被用于展示不同粒度的地图。它通常存在泛化\ \underline{精准度}和泛化\ \underline{速度}之间的权衡。
\\[2pt]
GENE-MAP提出了一种基于\ \underline{迭代}和\ \underline{贪心}的权衡算法,为地图泛化提供了一种解决方案。
\\[2pt]
GENE-MAP是个人项目,我完成的工作有:
\begin{itemize}[parsep=0.5ex]
  \item 设计并实现了一种基于迭代和贪心的泛化方法
  \item 利用OpenMP并行泛化方法
\end{itemize}

\datedsubsection{\textbf{TiDB}}{ 2015 - 至今 }
\role{分布式数据库}{Go,创业项目}
TiDB是PingCAP创业团队开发的分布式 SQL 数据库。
\\[2pt]
TiDB是开源项目,我参与的贡献有:
\begin{itemize}[parsep=0.5ex]
  \item 实现了数个兼容 MySQL 的内置函数
  \item 修复了已有内置函数实现上的一个缺陷
  \item 修复了部分文档的拼写和语法错误
\end{itemize}

\section{\faFile\ 论文发表}
\titleformat{\subsection}
  {\normalsize\raggedright}
  {}{0em}
  {}
\datedsubsection{\textbf{Fast and General Distributed Transactions using RDMA and HTM}}{\textbf{EuroSys} 2016}
\datedline{\small\underline{Yanzhe Chen}, Xinda Wei, Jiaxin Shi, Rong Chen and Haibo Chen.}{\href{}{\faFilePdfO}\ \ \href{}{\faFilePowerpointO}}

\datedsubsection{\textbf{Fast In-memory Transaction Processing using RDMA and HTM}}{\textbf{SOSP} 2015}
\datedline{\small Xingda Wei, Jiaxin Shi, \underline{Yanzhe Chen}, Rong Chen, Haibo Chen.}{\href{http://7xra08.com1.z0.glb.clouddn.com/pubs/drtm-sosp15.pdf}{\faFilePdfO}\ \ \href{http://7xra08.com1.z0.glb.clouddn.com/slides/drtm-sosp15-slides.pptx}{\faFilePowerpointO}}

\titlespacing*{\subsection}{0cm}{*1.8}{*0.6}
\datedsubsection{\textbf{PowerLyra: Differentiated Graph Computation and Partitioning on Skewed Graphs}}{\textbf{EuroSys} 2015}
\datedline{\small Rong Chen, Jiaxin Shi, \underline{Yanzhe Chen}, Haibo Chen.}{\href{http://7xra08.com1.z0.glb.clouddn.com/pubs/powerlyra-eurosys15.pdf}{\faFilePdfO}\ \ \href{http://7xra08.com1.z0.glb.clouddn.com/slides/powerlyra-eurosys15-slides.pptx}{\faFilePowerpointO}}

\datedsubsection{\textbf{Greedy Map Generalization by Iterative Point Removal}}{\textbf{SIGSPATIAL} 2014}
\datedline{\small\underline{Yanzhe Chen}, Yin Wang, Rong Chen, Haibo Chen and Binyu Zang.}{\href{http://7xra08.com1.z0.glb.clouddn.com/pubs/sigspatialcup14.pdf}{\faFilePdfO}\ \ \href{http://7xra08.com1.z0.glb.clouddn.com/slides/sigspatialcup14-slides.pptx}{\faFilePowerpointO}}

\section{\faHeartO\ 获奖情况}
\datedsubsection{ACM EuroSys 最佳论文奖}{2015}
\datedsubsection{上海交通大学硕士一等学业奖学金}{2014}
\datedsubsection{ACM SIGSPATIAL GISCUP 第三名}{2014}
\datedsubsection{上海交通大学优秀毕业生}{2014}
\datedsubsection{上海交通大学心动二等奖学金}{2013}
\datedsubsection{新鸿基地产郭氏基金奖助学金}{2012}
\datedsubsection{上海交通大学C等奖学金}{2012}
\datedsubsection{申银万国奖学金}{2011}

\section{\faBook\ 助教工作}
\datedsubsection{计算机系统设计和实现}{2016}
\datedsubsection{分布式系统}{2015}
\datedsubsection{程序设计}{2013}

\section{\faCogs\ 专业技能}
\begin{itemize}[parsep=1.0ex]
  \item 编程语言: 熟练使用 C\ /\ C++,了解 Python, Bash, Java, Go
  \item 英语水平: CET-4 633分,CET-6 561分,可用英语做展示和日常交流
  \item 系统工具: 4年以上的Unix使用经验,了解 Git,Vim,GCC 等开发工具
\end{itemize}

\end{document}
