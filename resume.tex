% !TEX program = xelatex

\documentclass{resume}
%\usepackage{zh_CN-Adobefonts_external} % Simplified Chinese Support using external fonts (./fonts/zh_CN-Adobe/)
%\usepackage{zh_CN-Adobefonts_internal} % Simplified Chinese Support using system fonts
\usepackage{hyperref}

\begin{document}
\pagenumbering{gobble} % suppress displaying page number

\name{Yanzhe Chen}

% {E-mail}{mobilephone}{homepage}
% be careful of _ in emaill address
\contactInfo{yanzhe.cn@gmail.com}{(+86) 185-1669-3610}{}
% {E-mail}{mobilephone}
% keep the last empty braces!
%\contactInfo{xxx@yuanbin.me}{(+86) 131-221-87xxx}{}

\section{\faGraduationCap\ Education}
\datedsubsection{\normalsize \textbf{Institute of Parallel and Distributed Systems, Shanghai Jiao Tong University}}{2014.9 - 2017.3}
\datedline{\small \textit{M.S. in Software Engineering, Advisor: \href{http://ipads.se.sjtu.edu.cn/binyu_zang}{Prof. Binyu Zang}}}{\textbf{GPA}: 2.81\ /\ 3.3, \textbf{Rank}: 2\ /\ 95}
\datedsubsection{\normalsize \textbf{Shanghai Jiao Tong University}}{2010.9 - 2014.6}
\datedline{\small \textit{B.Eng. in Software Engineering}}{\textbf{GPA}: 3.79\ /\ 4.3, \textbf{Rank}: 8\ /\ 99}

\section{\faBriefcase\ Internship}
\datedsubsection{\textbf{Microsoft, China}}{ 2016.7 - 2016.9 }
\role{Cloud \& Enterprise Group}{OSTC, SDE Intern}
Developed an Azure resource monitor system called Rocket. Rocket can track the usage status across different data centers and raise the alert when idle resources are found. With the help of the Rocket, team members can have a better understanding of the cluster usage and save money.
\\[5pt]
Rocket is built as a SPA using the MEAN stack. It can ship with the Docker and have the advantage of cross platform and lightweight deployment.

\section{\faUsers\ Projects}
\datedsubsection{\textbf{DrTM}}{ 2015 - 2016 }
\role{Distributed Transaction Processing}{C++, Research Project}
DrTM is a fast and general in-memory transaction system which provides high throughput and low latency.
\\[5pt]
DrTM leverages two hardware features called HTM and RDMA and designed a hardware-friendly protocol to boost the distributed transaction processing. Two papers based on this project were successively accepted by SOSP’15 and EuroSys’16.
\\[5pt]
DrTM is a \textit{cooperated} project, my contributions are:
\begin{itemize}
  \item {Design and implement an OCC concurrency control scheme. Enable transaction backup and recovery using HTM and RDMA.}
  \item {Implement a suite of benchmarks (TPC-C, SmallBank, TATP) that confirms the high performance.}
\end{itemize}

\datedsubsection{\textbf{PowerLyra}}{ 2013 - 2015 }
\role{Distributed Graph Computation}{C++, Research Project}
Graph computation is widely used to reason about large-scale complex data in machine learning and data mining. PowerLyra is a performance extension to GraphLab, which is a open source graph computation framework.
\\[5pt]
PowerLyra discusses two ways of graph partition: Vertex-Cut and Edge-Cut and analyses their merits and demerits. It shows a hybrid partition to reduce redundant message and improves the performance dramatically. The paper based on this project was accepted by EuroSys’15 and won the best paper award.
\\[5pt]
PowerLyra is a \textit{cooperated} project, my contributions are:
\begin{itemize}
  \item {Implement a \textit{hybrid} graph partition algorithm that leverages both locality and parallelism.}
\end{itemize}

\datedsubsection{\textbf{GENE-MAP}}{ 2013 - 2014 }
\role{Map Generalization}{C++, Contest Project}
Map generalization is one of the core technologies for online map service. GENE-MAP provides an efficient generalization algorithm to make a trade off between precision and speed.
\\[5pt]
GENE-MAP exploits \textit{iteration} and \textit{greedy} strategy to ensure a good balance between precision and speed. It won the third place in 2014 ACM GISCUP competition and its paper was accepted by SIGSPATIAL’14.
\\[5pt]
GENE-MAP is a \textit{personal} project, my contributions are:
\begin{itemize}
  \item {Design and implement an \textit{iterative} and \textit{greedy} algorithm for map generalization.}
  \item {Parallelize the algorithm using the OpenMP library.}
\end{itemize}

\datedsubsection{\textbf{TiDB}}{ 2015 - 2016 }
\role{Distributed Database}{Go, Open Source Practise}
TiDB is a distributed SQL database developed by PingCAP. I contributed 4 patches to the community.
\\[5pt]
TiDB is an \textit{open source} project, my contributions are:
\begin{itemize}
  \item {Implement two built-in functions in MySQL compatible layer.} (\href{https://github.com/pingcap/tidb/pull/755}{PR \#755})
  \item {Resolve an exception when handling invalid input in a built-in function.} (\href{https://github.com/pingcap/tidb/pull/759}{PR \#759})
  \item {Fix the connection id inconsistency problem.} (\href{https://github.com/pingcap/tidb/pull/770}{PR \#770})
  \item {Fix multiple typos and grammar errors in the documentation.} (\href{https://github.com/pingcap/tidb/pull/757}{PR \#757})
\end{itemize}

\section{\faFile\ Publications}
\titleformat{\subsection}
  {\normalsize\raggedright}
  {}{0em}
  {}
\datedsubsection{\textbf{Fast and General Distributed Transactions using RDMA and HTM}}{\textbf{EuroSys} 2016}
\datedline{\small\underline{Yanzhe Chen}, Xinda Wei, Jiaxin Shi, Rong Chen and Haibo Chen.}{\href{http://ob88vwut3.bkt.clouddn.com/papers/drtmr-eurosys16.pdf}{\faFilePdfO}\ \ \href{http://ob88vwut3.bkt.clouddn.com/slides/drtmr-eurosys16-slides.pptx}{\faFilePowerpointO}}

\datedsubsection{\textbf{Fast In-memory Transaction Processing using RDMA and HTM}}{\textbf{SOSP} 2015}
\datedline{\small Xingda Wei, Jiaxin Shi, \underline{Yanzhe Chen}, Rong Chen, Haibo Chen.}{\href{http://ob88vwut3.bkt.clouddn.com/papers/drtm-sosp15.pdf}{\faFilePdfO}\ \ \href{http://ob88vwut3.bkt.clouddn.com/slides/drtm-sosp15-slides.pptx}{\faFilePowerpointO}}

\titlespacing*{\subsection}{0cm}{*1.8}{*0.6}
\datedsubsection{\textbf{PowerLyra: Differentiated Graph Computation and Partitioning on Skewed Graphs}}{\textbf{EuroSys} 2015}
\datedline{\small Rong Chen, Jiaxin Shi, \underline{Yanzhe Chen}, Haibo Chen.}{\href{http://ob88vwut3.bkt.clouddn.com/papers/powerlyra-eurosys15.pdf}{\faFilePdfO}\ \ \href{http://ob88vwut3.bkt.clouddn.com/slides/powerlyra-eurosys15-slides.pptx}{\faFilePowerpointO}}

\datedsubsection{\textbf{Greedy Map Generalization by Iterative Point Removal}}{\textbf{SIGSPATIAL} 2014}
\datedline{\small\underline{Yanzhe Chen}, Yin Wang, Rong Chen, Haibo Chen and Binyu Zang.}{\href{http://ob88vwut3.bkt.clouddn.com/papers/gmap-sigspatialcup14.pdf}{\faFilePdfO}\ \ \href{http://ob88vwut3.bkt.clouddn.com/slides/gmap-sigspatialcup14-slides.pptx}{\faFilePowerpointO}}

\section{\faHeartO\ Honors and Awards}
\datedsubsection{ACM EuroSys Best Paper Award}{2015}
\datedsubsection{First-class Academic Scholarship for M.S., Shanghai Jiao Tong University}{2014}
\datedsubsection{ACM SIGSPATIAL GISCUP \nth{3} Place}{2014}
\datedsubsection{Outstanding College Graduate of Shanghai Jiao Tong University}{2014}
\datedsubsection{XinDong Scholarship (second-class) of Shanghai Jiao Tong University}{2013}
\datedsubsection{Sun Hung Kai Properties Scholarship}{2012}
\datedsubsection{Academic Excellence Scholarship (third-class) of Shanghai Jiao Tong University}{2012}
\datedsubsection{ShenYin and WanGuo Special Scholarship of Shanghai Jiao Tong University}{2011}

\section{\faBook\ Teaching Assistant}
\datedsubsection{Computer System Design and Implementation}{2016}
\datedsubsection{Distributed Systems}{2015}
\datedsubsection{Introduction to Programming}{2013}

\section{\faCogs\ SKILLS}
\begin{itemize}[parsep=1.0ex]
  \item Programming Languages: Familiar with C\ /\ C++; Some experience with JavaScript, Python, Bash, Java, and Go
  \item English: CET-4 633, CET-6 561
  \item Systems and Tools: 4 years experience with Unix; Familiar with Git, Vim and GCC toolchains.
\end{itemize}

\end{document}
